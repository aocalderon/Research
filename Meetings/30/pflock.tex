\documentclass{beamer}

\usefonttheme{professionalfonts} % using non standard fonts for beamer
\usefonttheme{serif} % default family is serif

\usepackage{enumitem}
\setitemize{label=\usebeamerfont*{itemize item}%
  \usebeamercolor[fg]{itemize item}
  \usebeamertemplate{itemize item}}

\usepackage{hyperref}
\usepackage{booktabs}
\usepackage{xfp}
\usepackage{graphicx}
\def\Put(#1,#2)#3{\leavevmode\makebox(0,0){\put(#1,#2){#3}}}
\usepackage{colortbl}
\usepackage{tikz}
\usepackage{amssymb}
\usepackage{enumerate}
\usepackage{arydshln}
\usepackage{algorithm}
\usepackage{algpseudocode}
\usepackage{subcaption} %to have subfigures available

\usepackage[absolute,overlay]{textpos}

\colorlet{lightred}{red!25}
\colorlet{lightgreen}{green!25}
\beamertemplatenavigationsymbolsempty

\newcommand\blfootnote[1]{%
  \begingroup
  \renewcommand\thefootnote{}\footnote{#1}%
  \addtocounter{footnote}{-1}%
  \endgroup
}

\makeatletter

%% Textclass specific LaTeX commands.
\newcommand\makebeamertitle{\frame{\maketitle}}%
\AtBeginDocument{%
  \let\origtableofcontents=\tableofcontents
  \def\tableofcontents{\@ifnextchar[{\origtableofcontents}{\gobbletableofcontents}}
  \def\gobbletableofcontents#1{\origtableofcontents}
}
%% User specified LaTeX commands.
\usetheme{Malmoe}
\useoutertheme{infolines}
\addtobeamertemplate{headline}{}{\vskip2pt}
\setbeamercovered{transparent}

\title[PFlock report]{PFLOCK Report}
\author[AC]{Andres Calderon}
\institute[UCR]{University of California, Riverside}
\makeatother

%%%%%%%%%%%%%%%%%%%%%%%%%%%%%%%%%%%%%%
%% Main document
%%%%%%%%%%%%%%%%%%%%%%%%%%%%%%%%%%%%%%
\begin{document}
\makebeamertitle
\newif\iflattersubsect

\AtBeginSection[] {
    \begin{frame}<beamer>
    \frametitle{Outline} 
    \tableofcontents[currentsection]  
    \end{frame}
    \lattersubsectfalse
}

\AtBeginSubsection[] {
    \begin{frame}<beamer>
    \frametitle{Outline} 
    \tableofcontents[currentsubsection]  
    \end{frame}
}

%%%%%%%%%%%%%%%%%%%%%%%%%%%%%%%%%%%%%%%%%%%
%%% Slides...
%%%%%%%%%%%%%%%%%%%%%%%%%%%%%%%%%%%%%%%%%%%

\begin{frame}{Temporal partitioning strategy}
    \centering
    \includegraphics[width=0.6\textwidth]{figures/space}
\end{frame}

\begin{frame}{Temporal partitioning strategy}
    \centering
    \includegraphics[width=0.6\textwidth]{figures/flocks}
\end{frame}

\begin{frame}{Temporal partitioning strategy}
    \centering
    \includegraphics[width=0.6\textwidth]{figures/partitions}
\end{frame}

\begin{frame}{Temporal partitioning strategy}
    \centering
    \includegraphics[width=0.6\textwidth]{figures/Q1}
\end{frame}
\begin{frame}{Temporal partitioning strategy}
    \centering
    \includegraphics[width=0.6\textwidth]{figures/Q2}
\end{frame}
\begin{frame}{Temporal partitioning strategy}
    \centering
    \includegraphics[width=0.6\textwidth]{figures/Q3}
\end{frame}
\begin{frame}{Temporal partitioning strategy}
    \centering
    \includegraphics[width=0.6\textwidth]{figures/Q4}
\end{frame}

\begin{frame}{Temporal partitioning strategy}
    \centering
    \includegraphics[width=0.6\textwidth]{figures/report}
\end{frame}

\begin{frame}{Temporal partitioning strategy}
    \centering
    \includegraphics[width=0.6\textwidth]{figures/reduce}
\end{frame}

\begin{frame}{Temporal partitioning strategy...}
    \begin{itemize}
        \item Any flock fully contained inside the cell (cube?) can be reported safely...
        \item Any flock that touch `risky' area must be keep it for further post-processing...
        \item We should follow a strategy to concatenate possible segments of the same flock cut them along different partitions...
    \end{itemize}
\end{frame}
%%%%%%%%%%%%%%%%%%%%%%%%%%%%%%%%%%%%%%%%%%%

\end{document}

