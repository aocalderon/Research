\documentclass{beamer}

\usefonttheme{professionalfonts} % using non standard fonts for beamer
\usefonttheme{serif} % default family is serif

\usepackage{hyperref}
%\usepackage{minted}
\usepackage{animate}
\usepackage{graphicx}
\def\Put(#1,#2)#3{\leavevmode\makebox(0,0){\put(#1,#2){#3}}}
\usepackage{colortbl}
\usepackage{tikz}
\usepackage{amssymb}
\usepackage{enumerate}
\usepackage{arydshln}
\usepackage{algorithm}
\usepackage{algpseudocode}

\colorlet{lightred}{red!25}
\colorlet{lightgreen}{green!25}


\newcommand\blfootnote[1]{%

  \begingroup

  \renewcommand\thefootnote{}\footnote{#1}%

  \addtocounter{footnote}{-1}%

  \endgroup

}

\makeatletter

%%%%%%%%%%%%%%%%%%%%%%%%%%%%%% Textclass specific LaTeX commands.

 % this default might be overridden by plain title style

 \newcommand\makebeamertitle{\frame{\maketitle}}%

 % (ERT) argument for the TOC

 \AtBeginDocument{%

   \let\origtableofcontents=\tableofcontents

   \def\tableofcontents{\@ifnextchar[{\origtableofcontents}{\gobbletableofcontents}}

   \def\gobbletableofcontents#1{\origtableofcontents}

 }

%%%%%%%%%%%%%%%%%%%%%%%%%%%%%% User specified LaTeX commands.

\usetheme{Malmoe}

% or ...

\useoutertheme{infolines}

\addtobeamertemplate{headline}{}{\vskip2pt}

\setbeamercovered{transparent}

% or whatever (possibly just delete it)

\makeatother

\begin{document}
\title[PFLOCK report]{PFLOCK Report}
\author[AC]{Andres Calderon}
\institute[Winter'20]{University of California, Riverside}
\makebeamertitle
\newif\iflattersubsect

\AtBeginSection[] {
    \begin{frame}<beamer>
    \frametitle{Outline} 
    \tableofcontents[currentsection]  
    \end{frame}
    \lattersubsectfalse
}

\AtBeginSubsection[] {
    \begin{frame}<beamer>
    \frametitle{Outline} 
    \tableofcontents[currentsubsection]  
    \end{frame}
}


\begin{frame}{Partition based join...}
    \begin{enumerate}
        \item Partition centers and points using the same set of global grids (Quadtree grids).
        \item At each global grid, take a sample of centers and build a local quadtree.
        \item For each point in a global grid:
        \begin{enumerate}
            \item compute the position of that point according to the grids defined by the local quadtree.
        \end{enumerate}
        \item For centers in the same global grid:
        \begin{enumerate}
            \item compute the position of each center according to local grids and manage possible replication.
        \end{enumerate}
        \item For each local grid, filter pairs which do not satisfy the distance condition.
    \end{enumerate}
\end{frame}

\begin{frame}{Experiment setup...}
    \begin{itemize}
        \item Dataset: LA\_50K (time interval 320).
        \item $\varepsilon=45$ and $\mu=3$.
        \item Global partitioning: Quadtree, number of partitions: 256 ($\approx 750$ grids).
        \item Number of nodes: 12, cores per node: 9 (108 total cores).
        \item Variying values of capacity (maximum items per grid) and fraction (size of the sample respect to \# of points in grid) for the local grids.
    \end{itemize}
\end{frame}

\begin{frame}{Results...}
    \centering
    \includegraphics[width=0.85\textwidth]{figures/ByCapacity}
\end{frame}

\begin{frame}{Results...}
    \centering
    \includegraphics[width=0.85\textwidth]{figures/ByFraction}
\end{frame}

\begin{frame}{What's next?}
    \begin{itemize}
        \item Optional:
        \begin{itemize}
            \item Avoid the sampling during local quadtree creation and querying.  
            \item Try KDB-Tree as local partitioner.
        \end{itemize}
        \item Continuing with index-based join and integrate with previous code.
    \end{itemize}
\end{frame}


\end{document}
