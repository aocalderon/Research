\documentclass{beamer}

\usefonttheme{professionalfonts} % using non standard fonts for beamer
\usefonttheme{serif} % default family is serif

\usepackage{hyperref}
%\usepackage{minted}
\usepackage{animate}
\usepackage{graphicx}
\def\Put(#1,#2)#3{\leavevmode\makebox(0,0){\put(#1,#2){#3}}}
\usepackage{color}
\usepackage{tikz}
\usepackage{amssymb}
\usepackage{enumerate}


\newcommand\blfootnote[1]{%

  \begingroup

  \renewcommand\thefootnote{}\footnote{#1}%

  \addtocounter{footnote}{-1}%

  \endgroup

}

\makeatletter

%%%%%%%%%%%%%%%%%%%%%%%%%%%%%% Textclass specific LaTeX commands.

 % this default might be overridden by plain title style

 \newcommand\makebeamertitle{\frame{\maketitle}}%

 % (ERT) argument for the TOC

 \AtBeginDocument{%

   \let\origtableofcontents=\tableofcontents

   \def\tableofcontents{\@ifnextchar[{\origtableofcontents}{\gobbletableofcontents}}

   \def\gobbletableofcontents#1{\origtableofcontents}

 }

%%%%%%%%%%%%%%%%%%%%%%%%%%%%%% User specified LaTeX commands.

\usetheme{Malmoe}

% or ...

\useoutertheme{infolines}

\addtobeamertemplate{headline}{}{\vskip2pt}

\setbeamercovered{transparent}

% or whatever (possibly just delete it)

\makeatother

\begin{document}
\title[PFLOCK report]{PFLOCK Report}
\author[AC]{Andres Calderon}
\institute[Fall'19]{University of California, Riverside}
\makebeamertitle
\newif\iflattersubsect

\AtBeginSection[] {
    \begin{frame}<beamer>
    \frametitle{Outline} 
    \tableofcontents[currentsection]  
    \end{frame}
    \lattersubsectfalse
}

\AtBeginSubsection[] {
    \begin{frame}<beamer>
    \frametitle{Outline} 
    \tableofcontents[currentsubsection]  
    \end{frame}
}

\begin{frame}{Working on Brinkhoff dataset}
    \begin{itemize}
        \item Double-checking some figures. Length of trajectories (in time instants):
        \item Original dataset: \\
        \begin{tabular}{|c|c|c|} 
            \hline
            avg & min & max \\ \hline
            2218.39 & 82 & 2665 \\ \hline
        \end{tabular}
        \item New dataset: \\
        \begin{tabular}{|c|c|c|} 
            \hline
            avg & min & max \\ \hline
            556.40 & 1 & 1134 \\ \hline
        \end{tabular}
        \item I have prepared some notebooks with additional computations...
    \end{itemize}
\end{frame}

\begin{frame}{Performing experiments in Brinkhoff dataset}
    {\small Sample 1: 100 Time instants from 0 to 100 ($\approx$360 points per time instant).}
    \centering
    \begin{figure}
        \includegraphics[width=.9\textwidth]{ICPE_B0-1K}
    \end{figure}    
\end{frame}

\begin{frame}{Performing experiments in Brinkhoff dataset}
    {\small Sample 2: 200 Time instants from 44K to 44.2K ($\approx$813 points per time instant).}
    \centering
    \begin{figure}
        \includegraphics[width=.9\textwidth]{ICPE_B43K-44K}
    \end{figure}    
\end{frame}

\begin{frame}{Revisiting LA dataset}
    \centering
    \begin{figure}
        \includegraphics[width=.9\textwidth]{ICPE_10K}
    \end{figure}    
\end{frame}

\begin{frame}{Revisiting LA dataset}
    \centering
    \begin{figure}
        \includegraphics[width=.9\textwidth]{ICPE_25K}
    \end{figure}    
\end{frame}

\begin{frame}{What is next?}
    \begin{itemize}
        \item I am still working on adapting the ID-based partititioning under the Spark Streaming environment. I have finished integrating the code but I am getting problems to coordinate the window operations and the ingestion of the data.
        
        \item Once it is fixed I expect to implement the Fixed Length Bit Compression method as proposed on Chen et al.
    \end{itemize}
\end{frame}

\end{document}
