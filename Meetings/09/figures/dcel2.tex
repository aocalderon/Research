\documentclass[border=2mm]{article}
\primef'\colon\usepackage{harpoon}
\usepackage{amsmath}
\usepackage{booktabs}
\usepackage{xcolor}
\usepackage{tikz}
\usetikzlibrary{arrows, arrows.meta}

\begin{document}
\tikzset{
    larrow/.style={ % style that just defines the arrow tip
        >={Triangle[left,length=6pt,width=4pt]},
        shorten >= 4pt, shorten <= 4pt,
        semithick,
        ->
    }
}
\newcommand{\halfedge}[2]
{
    \draw[larrow, transform canvas={yshift=1.5pt}] (#1) -- (#2);
    \draw[larrow, transform canvas={yshift=-1.5pt}] (#2) -- (#1);
}
\begin{tikzpicture}
    \tikzstyle{node1}=[draw,shape=circle,color=black,fill=white, minimum size=5pt]
    \tikzstyle{node2}=[color=black,fill=white]

    \node[node1] (A) at (0,2)   {a};
    \node[node1] (B) at (2,0)   {b};
    \node[node1] (C) at (2,4)   {c};
    \node[node1] (D) at (4,2)   {d};
    \node[node1] (E) at (4,6)   {e};
    \node[node1] (F) at (6,4)   {f};

    \halfedge{A}{C}
    \halfedge{A}{B}
    \halfedge{C}{D}
    \halfedge{B}{D}
    \draw[larrow, transform canvas={yshift=1.5pt}] (E) -- (F) node[pos=0.4, right] {$twin(\vec{fe})$};
    \draw[larrow, transform canvas={yshift=-1.5pt}] (F) -- (E) node[pos=0.4, left] {$\vec{fe}$};
    \draw[larrow, transform canvas={yshift=1.5pt}] (C) -- (E);
    \draw[larrow, transform canvas={yshift=-1.5pt}] (E) -- (C) node[pos=0.8, right] {$next(\vec{fe})$};
    \draw[larrow, transform canvas={yshift=1.5pt}] (D) -- (F) node[pos=0.5, left] {$prev(\vec{fe})$};
    \draw[larrow, transform canvas={yshift=-1.5pt}] (F) -- (D);
    \node[node2] (P) at (6,1) {$incidentFace(\vec{fe})$};
    \node[node2] (S) at (0.5,4) {$f_3$};
    \node[node2] (Q) at (4,4) {$f_2$};
    \node[node2] (R) at (2,2) {$f_1$};
    \draw[->, red] (P) to [out=90,in=0] (Q);

\end{tikzpicture}


\begin{table}
    \centering
    \begin{tabular}{c c c}
        \toprule
        vertex & coordinates & incident edge \\
        \midrule
        \overrightharp{ab} & (0,2) & $\vec{ab}$ \\
        b & (2,0) & $\vec{bd}$ \\
        c & (2,4) & $\vec{cd}$ \\
        d & (4,2) & $\vec{df}$ \\
        e & (4,6) & $\vec{ef}$ \\
        f  & (6,4) & $\vec{fd}$ \\
        \bottomrule
    \end{tabular}
\end{table}

\end{document}
