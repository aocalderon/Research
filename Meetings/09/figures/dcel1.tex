\documentclass[border=0cm]{standalone}
\usepackage{xcolor}
\usepackage{tikz}
\usetikzlibrary{arrows, arrows.meta}
\tikzset{
    barbarrow/.style={ % style that just defines the arrow tip
        >={Straight Barb[left,length=5pt,width=5pt]},
        thick,
        ->
    },
    blues/.style={
        color=blue
    },
    reds/.style={
        color=red
    }
}
\definecolor{light-gray}{gray}{0.975}
\definecolor{pcolor}{rgb}{0.21, 0.27, 0.31}
\begin{document}
\begin{tikzpicture}
    \tikzstyle{node1}=[draw,scale=0.4,shape=circle,color=black,fill=black]
    \tikzstyle{node2}=[draw,scale=0.4,shape=circle,color=red,fill=red]
    \tikzstyle{text}=[draw,scale=0.5,color=black]


    \node[node1] (a) at (3,6) {};
    \node[node1] (b) at (6,6) {};
    \node[node1] (c) at (2,3) {};
    \node[node1] (d) at (7,2) {};
    \node at (2,2) {$A$};

    \draw (0,0) -- (0,8);\draw (0,0) -- (8,0);
    \draw (a) -- (b);
    \draw (b) -- (d); %\draw[barbarrow, color=red] (E.-155) -- (A.25);
    \draw (d) -- (c); %\draw[barbarrow, color=red] (K.-115) -- (C.25);
    \draw (c) -- (a); %\draw[barbarrow, color=red] (K.155) -- (E.-65);
    \draw[barbarrow, color=red] (6.7, 2.2) -- (5.8, 5.8); %\draw[barbarrow, color=red] (C.155) -- (A.-65);
    \draw[barbarrow, color=red] (5.8, 5.8) -- (3.2, 5.8); %\draw[barbarrow, color=red] (C.155) -- (A.-65);
    \draw[barbarrow, color=red] (3.2, 6.2) -- (5.8, 6.2); %\draw[barbarrow, color=red] (C.155) -- (A.-65);
    \draw[barbarrow, color=red] (3.2, 5.8) -- (2.3, 3.2); %\draw[barbarrow, color=red] (C.155) -- (A.-65);

\end{tikzpicture}
\end{document}
