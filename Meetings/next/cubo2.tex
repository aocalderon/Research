\documentclass[border=0.25cm]{standalone}
\usepackage{tikz}
\usepackage{xcolor}
\tikzset{
    point/.style={  }
}

\begin{document}
\newcommand{\Depth}{2}
\newcommand{\Height}{2}
\newcommand{\Width}{3.5}
\begin{tikzpicture}
\coordinate (O) at (0,0,0);
\coordinate (X) at (5,0,0);
\coordinate (Y) at (0,0,4);
\coordinate (Z) at (0,4,0);
\draw[-stealth] (O) -- (X);
\draw[-stealth] (O) -- (Y);
\draw[-stealth] (O) -- (Z);
\node[] at (4.5,0.15,0) {x};
\node[] at (0,0.15,3) {y};
\node[] at (0.12,3.75,0) {t};

\coordinate (A) at (0,\Width,0);
\coordinate (B) at (0,\Width,\Height);
\coordinate (C) at (0,0,\Height);
\coordinate (D) at (\Depth,0,0);
\coordinate (E) at (\Depth,\Width,0);
\coordinate (F) at (\Depth,\Width,\Height);
\coordinate (G) at (\Depth,0,\Height);

\draw[] (O) -- (C) -- (G) -- (D) -- cycle;% Bottom Face
\draw[] (O) -- (A) -- (E) -- (D) -- cycle;% Back Face
\draw[] (O) -- (A) -- (B) -- (C) -- cycle;% Left Face
\draw[fill=black!20,opacity=0.8] (A) -- (B) -- (F) -- (E) -- cycle;% Top Face
\draw[fill=black!20,opacity=0.8] (D) -- (E) -- (F) -- (G) -- cycle;% Right Face
\draw[fill=black!20,opacity=0.8] (C) -- (B) -- (F) -- (G) -- cycle;% Front Face

\coordinate (AO) at (2.1,0,0);
\coordinate (A1) at (2.1,\Width,0);
\coordinate (B1) at (2.1,\Width,\Height);
\coordinate (C1) at (2.1,0,\Height);
\coordinate (D1) at (2.1+\Depth,0,0);
\coordinate (E1) at (2.1+\Depth,\Width,0);
\coordinate (F1) at (2.1+\Depth,\Width,\Height);
\coordinate (G1) at (2.1+\Depth,0,\Height);
\draw[] (AO) -- (C1) -- (G1) -- (D1) -- cycle;% Bottom Face
\draw[] (AO) -- (A1) -- (E1) -- (D1) -- cycle;% Back Face
\draw[] (AO) -- (A1) -- (B1) -- (C1) -- cycle;% Left Face
\draw[fill=black!20,opacity=0.8] (A1) -- (B1) -- (F1) -- (E1) -- cycle;% Top Face
\draw[fill=black!20,opacity=0.8] (D1) -- (E1) -- (F1) -- (G1) -- cycle;% Right Face
\draw[fill=black!20,opacity=0.8] (C1) -- (B1) -- (F1) -- (G1) -- cycle;% Front Face

\coordinate (P0) at (1,0,1); \draw[point] (P0) circle (0.075);
\coordinate (P1) at (3,0.25,1); \draw[point] (P1) circle (0.075);
\coordinate (P2) at (3.25,1.25,1); \draw[point] (P2) circle (0.075);
\coordinate (P3) at (1.25,1.5,1); \draw[point] (P3) circle (0.075);
\coordinate (P4) at (0.75,2.15,1); \draw[point] (P4) circle (0.075);
\coordinate (P5) at (1.25,2.55,1); \draw[point] (P5) circle (0.075);
\coordinate (P6) at (3.5,2.75,1); \draw[point] (P6) circle (0.075);

\draw[dashed] (1,0.075,1) -- (3,0.25+0.075,1);
\draw[dashed] (1,-0.075,1) -- (3,0.25-0.075,1);

\draw[dashed] (3+0.075,0.25,1) -- (3.25+0.075,1.23,1);
\draw[dashed] (3-0.075,0.25,1) -- (3.25-0.075,1.23,1);

\draw[dashed] (3.25,1.25+0.075,1) -- (1.25,1.5+0.075,1);
\draw[dashed] (3.25,1.25-0.075,1) -- (1.25,1.5-0.075,1);

\draw[dashed] (1.25+0.0275,1.5+0.075,1) -- (0.75+0.06,2.20,1);
\draw[dashed] (1.25-0.058,1.5-0.05,1) -- (0.755-0.08,2.13,1);

\draw[dashed] (0.75+0.075,2.15,1) -- (1.25,2.55-0.075,1);
\draw[dashed] (0.75-0.05,2.15+0.06,1) -- (1.25-0.05,2.55+0.06,1);

\draw[dashed] (1.25,2.55+0.075,1) -- (3.5,2.75+0.075,1);
\draw[dashed] (1.25,2.55-0.075,1) -- (3.5,2.75-0.075,1);

%% Following is for debugging purposes so you can see where the points are
%% These are last so that they show up on top
%\foreach \xy in {O, A, B, C, D, E, F, G}{
%    \node at (\xy) {\xy};
%}
\end{tikzpicture}
\end{document}
